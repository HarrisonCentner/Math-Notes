\documentclass[12pt]{article}
\usepackage{titlesec}
\usepackage{titling}
\usepackage[margin=.75in]{geometry}
\usepackage{pagecolor,lipsum}
\usepackage{xcolor}
\usepackage{fancyhdr}
\usepackage{multicol}
\usepackage{lmodern}
\pagecolor{white}
\usepackage{enumitem}
\usepackage{lineno}
\usepackage{parskip}
\usepackage{amssymb}
\usepackage{graphicx}
\usepackage{geometry}
\geometry{letterpaper}
\usepackage{amsmath}
\usepackage{mathrsfs}
\usepackage{array}
\usepackage{tcolorbox}

\linespread{1}

\pagestyle{fancy}

\newcommand*{\vertbar}{\rule[-1ex]{0.5pt}{2.5ex}}
\newcommand*{\horzbar}{\rule[.5ex]{2.5ex}{0.5pt}}

\titleformat{\section}%[frame]
{\vspace{-.15in}\LARGE\flushleft} %formatting
{} %numbering {\thesection}
{0em} %distance between number and section title
{\bfseries \uppercase}[\titlerule ] %Any code you want after gap before title

\titleformat{\subsection}
{\vspace{-1em}\Large\bfseries}
{\hspace{0em}}%$\bullet$}
{0em}
{\vspace{.3em}}[\vspace{-1.2em}]


\titleformat{\subsubsection}[runin] %runin eliminates in dent
{\vspace{0em}\bfseries}
{}
{0em}
{}[\ --- ]

\titlespacing{\subsubsection}
{0em}{-2em}{.5em} %Spacing of argument 1 left margin. 2 space before title. 3 space after title

\fancyhead{}

\renewcommand{\maketitle}{%renames a command
\begin {flushleft}{\fontsize{25pt}{33pt}\bfseries
\theauthor}}



% SHORTCUT COMMANDS
\newcommand{\p}{{\it{Proof.}}}
\newcommand{\z}{\vec{0}}
\newcommand{\R}{\mathbb{R}}
\newcommand{\qed}{\hfill \begin{flushright}$\square$\end{flushright}}
\newcommand{\B}{\mathcal{B}}
\newcommand{\hf}{\hfill}





%\rfoot{I made this r\'esum\'e with {\LaTeX}!}

\rhead{Introduction to Combinatorics — Math 465}
\renewcommand{\headrule}{\dotfill}


\begin{document}

\author{Graph Theory Exam Notes}
\date{\today}
\fancyhead{}
\fancyhead[RO,RE]{Harry Centner — Prof. Terrence George — Exam One}

% \setbeamersize{text margin top=10mm}
\maketitle

\vspace{.7em} % adds a vertical space

\large{hcentner@umich.edu \hfill Harry Centner} \\
\hfill Professor Terrence George 


\end{flushleft}

\vspace{-1em}

\section{Graph Basics}
\begin{center} \end{center}
\vspace{-1em}

	\subsection{Definitions}
	\begin{center} \end{center}
	\begin{tcolorbox}
	\textbf{Simple Graph} \\
	\null \quad A simple graph is one with no multiple edges or loops. \\ 
	
	\textbf{Walk} \\
	\null \quad A walk is an ordered set of vertices in which any two contiguous vertices are adjacent.  \\
	
	\textbf{Path} \\
	\null \quad A path is a walk with no repeated vertices.  \\
	\null \quad  $\{$paths of $G$$\} \subseteq \{$walks of $G\}$. \\
	
	\textbf{Clique} \\
	\null \quad A clique is a subset of vertices in which any two vertices are non-adjacent. Equivalently, \\
	\null \quad a clique is a $K_n$ subgraph of $G$. The clique number $\omega(G)$ is the largest clique size in $G$. \\ 
	
	\textbf{Independent Set} \\
	\null \quad  An independent set is a subset of vertices such that no two vertices are adjacent. \\
	\null \quad  The largest cardinality of an independent set in $G$ is denoted $\alpha(G)$ \\
	
	
	\textbf{Vertex Cover} \\
	\null \quad A vertex cover is a subset of vertices that contains at least one endpoint of every edge. \\
	\null \quad  The minimum size of a vertex cover is $$|V| - \alpha(G)$$
	\null \quad For any connected graph let $\rho(G)$ be the size of a minimum edge cover. Then
	$$\rho(G) + \alpha(G) = |V|$$

	\end{tcolorbox}
	
	
	\subsection{The First Theorem of Graph Theory}
	\begin{center} \end{center}
	\begin{tcolorbox}
	\text{\textbf{Binomial Definitions }  }\\
	\text{Let $G =(V,E)$ be a graph. Then,}
	
	\hfill
	\begin{center}
	$\begin{aligned}
	\sum_{v \in V} \deg(v) = 2|E|
	\end{aligned}$
	\end{center}
	\end{tcolorbox}
	
	
	
	
	
\newpage
\section{Planarity}
\begin{center} \end{center}
\vspace{-1em}
	
	\subsection{Graph Characteristics}
	\begin{center} \end{center}
	\begin{tcolorbox}
	\textbf{Zeroth Betti Number} \\
	\null \quad denoted $b_0 = b_0(G)$ is the number of connected components of a graph. \\ 
	
	\textbf{First Betti Number} \\
	\null \quad denoted $b_1= b_1(G)$ is equivalent to any of the following: 
			\begin{itemize}
			\item the maximal number of edges whose removal does not increase $b_0$
			\item the number of edge deletions required to make $G$ a tree
			\item the number of independent cycles in a graph \\
			\end{itemize} 
	\textbf{Other Betti Numbers} \\
	\null \quad All other Betti Numbers of a graph are zero. $b_i(G) = 0 \quad \forall i \in \mathbb{N} \setminus \{0,1\}$
	\end{tcolorbox}
	

	\begin{tcolorbox}
	\textbf{Euler Characteristic}\\
	\text{The Euler Characteristic of $G =(V,E)$ is given by}
	
	\hfill
	\begin{center}
	$\begin{aligned}
	|V| - |E| = b_0 - b_1
	\end{aligned}$
	\end{center}
	
	It follows that any three of the following imply the remaining statement:
	
	\begin{enumerate}[label = (\roman*)]
	\item $G$ is connected \hfill $b_0 =1$
	\item $G$ is acyclic \hfill $b_1 =0$
	\item $G$ has $n$ vertices \hfill $|V| =n$
	\item $G$ has $n-1$ edges \hfill $|E| = n-1$
	\end{enumerate}
	
	\end{tcolorbox}
	
	
	\subsection{Famous Theorems}
	\begin{center} \end{center}
	\begin{tcolorbox}
	\textbf{Euler's Formula}\\
	\text{The Euler Characteristic of $G =(V,E)$ is given by}
	
	\hfill
	\begin{center}
	$\begin{aligned}
	|V| - |E| + |F| &= b_0 + 1 \quad \text{(Planar Graph)} \\
	|V| - |E| + |F| &= 2 \quad \text{(Planar, Connected Graph)}
	\end{aligned}$ 
	\end{center}
	\end{tcolorbox}
	
	
	\begin{tcolorbox}
	\textbf{Corollaries}\\
	The following inequality is true for all connected planar graphs:
	$$|E| \le 3|V| - 6$$
	The following inequality is true for all connected, triangle free planar graphs with $|V| \ge 3$:
	$$|E| \le 2|V| -4$$
	The folllowing inequality is true for the 1-skeleton of any polyhedra:
	$$2|E| \ge 3|F|$$
	\end{tcolorbox}
	
	
	\begin{tcolorbox}
	\textbf{Kuratowski's Theorem}\\
	\text{Let $G =(V,E)$ be a simple graph, then}
	
	\hfill
	\begin{center}
	$G$ is planar is and only if $G$ contains no subgraph\\—or can be subdivided to be—
	isomorphic to $K_5$ or $K_{3,3}$.
	\end{center}
	\end{tcolorbox}
	
	
	
\section{Special Walks}
\begin{center} \end{center}
	
	\subsection{Hamiltonicity}
	\begin{center} \end{center}
	
	\begin{tcolorbox}
	\text{\textbf{Definition}  }\\
	\text{Let $G =(V,E)$ be a graph. Then,}
	
	\hfill
	\begin{center}
	$\begin{aligned}
	K_n \text{ has } \frac{(n-1)!}{2}& \text{ Hamiltonian cycles}\\
	 \text{A Hamiltonian graph} &\text{ has at least } {n-1 \choose 2} + 1  \text{ edges.}  
	\end{aligned}$
	\end{center}
	\end{tcolorbox}
	
	\begin{tcolorbox}
	\text{\textbf{Grinberg's Theorem}  }\\
	Let $G =(V,E)$ be a planar graph. Let $C$ be a Hamiltonian cycle in $G$. Let $F_1$ denote the set of faces inside $C$, and $F_2$ outside. For a face $f \in F = F_1 \cup F_2$, let $b(f)$ denote the number of edges bounding $f$. Then,
	$$\sum_{f \in F_1}(b(f) -2) = \sum_{f\in F_2}(b(f) -2) = |V|-2$$
	
	\end{tcolorbox}
	
	
	
	\begin{tcolorbox}
	\text{\textbf{Gray Codes}  }\\
	A $k$-digit Gray Code is a Hamiltonian cycle in $Q_k$ the hypercube graph.
	This corresponds to all $k$-digit binary strings listed cyclically such that contiguous strings differ by only one bit.
	\end{tcolorbox}
	
	
	
	\subsection{Eulerian Walks}
	\begin{center} \end{center}
	
	\begin{tcolorbox}
	\text{\textbf{Necessary Conditions for Graphs}  }\\
	A connected graph $G$ has an Eulerian walk if and only if: \\
	\begin{center} every vertex in $G$ has even degree \\ \quad OR \quad \\G has at most two vertices of odd degree  \end{center}
	
	\hfill
	
	A connected graph $G$ is Eulerian (has a closed trail) if and only if:
	\begin{center}  every vertex in $G$ has even degree \end{center}
	\end{tcolorbox}
	
	\begin{tcolorbox}
	\text{\textbf{Necessary Conditions for Digraphs}  }\\
	A digraph $G = (V,E)$ without isolated vertices the following are equivalent: 
	\begin{itemize}
	 \item $G$ has a closed Eulerian walk (Each directed edge can be traversed exactly once)
	 \item $G$ is connected (disregarding orientation) and indeg($v) =$ outdeg($v)$ for all $v \in V$
	 \item For any vertices $u,v \in V$ there is a walk that starts at $u$ and ends at $v$ and\\
	  indeg($v) =$ outdeg($v)$ for all $v \in V$
	 \end{itemize}
	\end{tcolorbox}
	
	
\section{Colorings}
\begin{center} \end{center}
\vspace{-1em}

	\subsection{Definitions}
	\begin{center} \end{center}
	
	
	\begin{tcolorbox}
	\textbf{Proper Coloring} \\
	\null \quad A proper coloring (coloring) of $G$ is a labelling of its vertices such that \\
	\null \quad  all adjacent vertices have distinct labels. It follows that for any graph $G(V,E)$,\\ 
	$$|E| \ge {\mathcal{X}(G)  \choose 2}$$
	
	\textbf{Chromatic Number} \\
	\null \quad The chromatic number of a graph $\mathcal{X}(G)$ is the least number of colors needed to color $G$. \\

	\end{tcolorbox}
	
	
	\subsection{Bounds on the Chromatic Number}
	\begin{center} \end{center}
	
	\begin{tcolorbox}
	\text{\textbf{Lower Bounds}  }\\
	Let $G =(V,E)$ be a simple graph, let $\omega(G)$ be the clique number of $G$, and \\
	let $\alpha(G)$ be the independence number of $G$. Then,
	
	\hfill
	\begin{center}
	$\begin{aligned}
	\mathcal{X}(G) &\ge \omega(G) \\
	\mathcal{X}(G) &\ge \frac{|V|}{\alpha(G)}\\
	\end{aligned}$
	\end{center}
	\end{tcolorbox}
	
	
	
	\begin{tcolorbox}
	\text{\textbf{Upper Bounds}  }\\
	\text{Let $G =(V,E)$ be a graph which is neither complete nor an odd cycle. Then,}
	
	\hfill
	\begin{center}
	$\begin{aligned}
	\mathcal{X}(G)  &\le \max_{v \in V} \deg(v) \\
	\end{aligned}$
	\end{center}
	\end{tcolorbox}
	
	
	\subsection{Chromatic Polynomials}
	\begin{center} \end{center}
	
	\begin{tcolorbox}
	\text{\textbf{Definition}  }\\
	Let $G$ be a graph. Then $p_G: \mathbb{Z}_\ge \rightarrow \mathbb{Z}$ is defined by:
	
	\hfill
	\begin{center}
	$\begin{aligned}
	p_G(k) = \text{ the number of colorings of $G$ with $k$ colors }
	\end{aligned}$
	\end{center}
	
	\text{\textbf{Consequences}  }
	\begin{enumerate}[label = (\roman*)]
	\item The coefficient of $k^{n-1}$ in $p_G(k)$ is $-|E|$. 
	\item $p_G(1) = 0$ if and only if $G$ has no edges.
	\item $\mathcal{X}(G) = \min\{ k \ | \ p_G(k) \ne 0\}$
	\end{enumerate}
	\end{tcolorbox}
	
	\begin{tcolorbox}
	\text{\textbf{Examples}  }\\
	
	\hfill
	$\begin{aligned}
	&\text{If $G$ is a forest with $c$ connected components, then} \quad &p_G(k) &=& k^c&(k-1)^{n-c} \\
	&\text{If $G = C_4$, then} \quad &p_{C_4}(k) &=& k&(k-1)Jk^2-3k+3) \\
	&\text{If $G$ has no edges, then} \quad &p_{C_4}(k) &=& k&^n \\
	\end{aligned}$
	\end{tcolorbox}
	
	\begin{tcolorbox}
	\text{\textbf{Recursive Formulas}  }\\
	Let $G =(V,E)$ be a simple graph and $e \in E$ an edge. Then, \\
	$G-e$ is the graph obtained from $G$ by deleting $e$.\\
	$G/e$ is the graph obtained from $G$ by contracting $e$.\\
	\hfill
	\begin{center}
	$\begin{aligned}
	p_G(k) = p_{G-e}(k) - p_{G/e}(k)
	\end{aligned}$
	\end{center}
	
	\hfill
	
	Let $G \sqcup H$ be the disjoint union of two graphs $H$ and $G$. Then,
	$$p_{G \sqcup H}(k) =p_G(k)  \cdot p_H(k)$$ 
	
	\hfill
	
	Let $G + H$ be the join of two graphs $H$ and $G$. Then,
	$$p_{G +H}(k) = \sum_a \sum_b \hat{p}_{G}(a)\hat{p}_H(b) {a+b \choose a} {k \choose a+b}$$
	\end{tcolorbox}
	
	\begin{tcolorbox}
	\text{\textbf{Using Exactly $k$ Colors}  }\\
	Let $G =(V,E)$ be a simple graph and let $\hat{p}_G(k)$ denote the number of proper colorings of $G$ using exactly $k$ colors.
	\hfill
	\begin{center}
	$\begin{aligned}
	p_G(k) = \sum_{a= 0}^{|V|} \hat{p}_G(k) { k \choose a}
	\end{aligned}$
	\end{center}
	\hfill note that $\hat{p}_G(k) \ne 0 \implies \mathcal{X}(G) \le a \le |V|$
	
	Note that $ \hat{p}_G(k)$ is the leftmost entry in the $a$-th row \\ of the difference table for the sequence $b = p_G(0), p_G(1), \dots$
	\end{tcolorbox}
	
	
	

\section{Digraphs \& Flows}
\begin{center} \end{center}
\vspace{-1em}

	\subsection{Definitions}
	\begin{center} \end{center}
	
	\begin{tcolorbox}
	\text{\textbf{Transportation Network}}\\
	\text{Let $G =(V,E)$ be a graph. Then, }

	\end{tcolorbox}
	
	
	\begin{tcolorbox}
	\text{\textbf{Flow}}\\
	\text{A flow in a transportation network is a function $f: E\rightarrow \R$ that satisfies two conditions,}
	\begin{enumerate}
	\item feasibility: $$0 \le f(e) \le c(e) \quad \forall e\in E$$
	\item conservatiton law: $$\sum_{\bullet\xrightarrow{e}v}f(e) = \sum_{v\xrightarrow{e}\bullet}f(e) \quad \forall v\in V \setminus\{s,t\} $$
	\end{enumerate}
	\end{tcolorbox}



	\begin{tcolorbox}
	\text{\textbf{Max-Flow Min-Cut Theorem}  }
	\begin{center}
	The maximum value of a flow in a transportation network\\
	 is equal to the minimum capacity of a cut:
	 \end{center}
	 
	$$\max_f |F| = \min_{(X,Y)} c(X,Y)$$
	\end{tcolorbox}
	
	
	\subsection{Matchings}
	\begin{center} \end{center}
	
	
	
	
\section{Spicy Extras}
\begin{center} \end{center}
\vspace{-1em}

	\subsection{Pigeon Hole Principle}
	\begin{center} \end{center}
	
	\begin{tcolorbox}
	\text{\textbf{Statement}  }
	\begin{center}
	If $|A| > |B|$, there are no injections $f: A \rightarrow B$.
	 \end{center}
	\end{tcolorbox}
	
	
	\begin{tcolorbox}
	\text{\textbf{Corollaries}  }
	\begin{enumerate}[label = (\roman*)]
	\item Congruence classes
	\item 
	\end{enumerate}
	\end{tcolorbox}
	
	
	
	\subsection{Posets}
	\begin{center} \end{center}
	
	\begin{tcolorbox}
	\text{\textbf{Definition}  }\\
	A partially ordered set (or poset) $P$ is a set endowed with a binary relation denoted $\le$, that satisfies three conditions:
	\begin{enumerate}[label = (\roman*)]
	\item reflexivity: $$\forall x\in P, \quad x \le x$$
	\item antisymmetricity: $$x \le y \wedge y \le x \implies x = y$$
	\item transitivity: $$x \le y \wedge y \le z \implies x \le z$$
	\end{enumerate}
	\end{tcolorbox}

	
	\subsection{Combinatorial Reciprocity Theorems}
	\begin{center} \end{center}
	
	\begin{tcolorbox}
	\text{\textbf{Acyclic Orientations}}\\
	The number of directed graphs obtained by orienting the edges of $G =(V,E)$ \\
	without creating a directed cycle is
	$$ao(G) = (-1)^{|V|} \cdot p_G(-1)$$
	And defined by the recursion
	$$ao(G) = ao(G-e) + ao(G/e)$$
	
	Note that if $G$ is a tree, then $ao(G) = 2^{|E|}$ \\
	 If $G = C_n$ is a cycle, then $ao(G) = 2^{n} -2$ \\
	 If $G = K_n$, then $ao(G) = n!$ \\
	\end{tcolorbox}
	
	\subsection{Automorphisms}
	\begin{center} \end{center}
	
	\begin{tcolorbox}
	\text{\textbf{Acyclic Orientations}}\\
	$K_n$ 
	\end{tcolorbox}

	
	
\end{document}