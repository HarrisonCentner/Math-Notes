\documentclass[12pt]{article}
\usepackage{titlesec}
\usepackage{titling}
\usepackage[margin=.75in]{geometry}
\usepackage{pagecolor,lipsum}
\usepackage{xcolor}
\usepackage{fancyhdr}
\usepackage{multicol}
\usepackage{lmodern}
\pagecolor{white}
\usepackage{enumitem}
\usepackage{lineno}
\usepackage{parskip}
\usepackage{amssymb}
\usepackage{graphicx}
\usepackage{geometry}
\geometry{letterpaper}
\usepackage{amsmath}
\usepackage{mathrsfs}
\usepackage{array}
\usepackage{tcolorbox}

\linespread{1}

\pagestyle{fancy}

\newcommand*{\vertbar}{\rule[-1ex]{0.5pt}{2.5ex}}
\newcommand*{\horzbar}{\rule[.5ex]{2.5ex}{0.5pt}}

\titleformat{\section}%[frame]
{\vspace{-.15in}\LARGE\flushleft} %formatting
{} %numbering {\thesection}
{.0em} %distance between number and section title
{\bfseries \uppercase}[\titlerule] %Any code you want after gap before title

\titleformat{\subsection}
{\vspace{-.4em}\Large\bfseries}
{\hspace{0em}}%$\bullet$}
{0em}
{\vspace{.3em}}[\vspace{-1.2em}]


\titleformat{\subsubsection}[runin] %runin eliminates in dent
{\vspace{.1em}\bfseries}
{}
{0em}
{}[\ --- ]

\titlespacing{\subsubsection}
{0em}{-2em}{.5em} %Spacing of argument 1 left margin. 2 space before title. 3 space after title

\fancyhead{}

\renewcommand{\maketitle}{%renames a command
\begin {flushleft}{\fontsize{25pt}{33pt}\bfseries
\theauthor}}



% SHORTCUT COMMANDS
\newcommand{\p}{{\it{Proof.}}}
\newcommand{\z}{\vec{0}}
\newcommand{\R}{\mathbb{R}}
\newcommand{\qed}{\hfill \begin{flushright}$\square$\end{flushright}}
\newcommand{\B}{\mathcal{B}}
\newcommand{\hf}{\hfill}





%\rfoot{I made this r\'esum\'e with {\LaTeX}!}

\rhead{Linear Algebra — Math 217}
\renewcommand{\headrule}{\dotfill}


\begin{document}

\author{Math 217 Exam Two Notes}
\date{\today}
\fancyhead{}
\fancyhead[RO,RE]{Harry Centner — Prof. Karen Smith 009 — Exam Two}

% \setbeamersize{text margin top=10mm}
\maketitle

\vspace{.7em} % adds a vertical space

\large{hcentner@umich.edu \hfill Harry Centner} \\
\hfill Professor Karen Smith 009


\end{flushleft}

\vspace{-2em}

\section{Coordinates}
\begin{center} \end{center}
	\subsection{Coordinates Matrices}
	\begin{center} \end{center}
	
	\begin{tcolorbox}
	\text{\textbf{Book Theorem 4.3.2}  }\\
	\text{The $\mathfrak{B}$-matrix of $T$} 
	\begin{center}
	$\begin{aligned}
	\text{[\,}T\text{]\,}_\mathfrak{B} &=  \begin{bmatrix}  |&| & & | \\ [\,T(\vec{v}_1)]\,_\mathfrak{B} & [\,T(\vec{v}_2)]\,_\mathfrak{B}  & \cdots & [\,T(\vec{v}_n)]\,_\mathfrak{B}  \\ |&|& & |\end{bmatrix} \\
	[\,T]\,_\mathfrak{B} &= S_{\mathfrak{U}\rightarrow\mathfrak{B}} [\,T]\,_\mathfrak{U} S_{\mathfrak{B}\rightarrow\mathfrak{U}}  
	\end{aligned}$
	\end{center}
	\end{tcolorbox}
	
	\begin{tcolorbox}
	\text{\textbf{Book Theorem 4.3.2}  }\\
	\text{Change of Basis Matrix $S_{\mathfrak{U}\rightarrow\mathfrak{B}}$}
	
	\hfill
	\begin{center}
	$\begin{aligned}
	S_{\mathfrak{U}\rightarrow\mathfrak{B}} & = (S_{\mathfrak{B}\rightarrow\mathfrak{U}})^{-1} \\
	S_{\mathfrak{U}\rightarrow\mathfrak{B}} &=  \begin{bmatrix}  |&| & & | \\ [\, \vec{u}_1]\,_\mathfrak{B} & [\,\vec{u}_2]\,_\mathfrak{B}  & \cdots & [\,\vec{u}_n]\,_\mathfrak{B}  \\ |&|& & |\end{bmatrix} \\
	\end{aligned}$
	\end{center}
	\end{tcolorbox}
	
	\begin{tcolorbox}
	\text{\textbf{Book Theorem 4.3.4}  }\\
	\text{Change of Basis in a subspace of $\R^n$}
	
	\hfill
	\begin{center}
	$\begin{aligned}
	  \begin{bmatrix}  |&| & &| \\ \vec{b}_1 & \vec{b}_2 & \cdots & \vec{b}_n \\   |&| & &| \end{bmatrix}=\begin{bmatrix}  |&| & &| \\ \vec{u}_1 & \vec{u}_2 & \cdots & \vec{u}_n \\   |&| & &| \end{bmatrix}S_{\mathfrak{B}\rightarrow\mathfrak{U}} \\
	\end{aligned}$
	\end{center}
	\end{tcolorbox}
	
	\subsection{Coordinate Theorems}
	\begin{center} \end{center}
	
	
	\begin{tcolorbox}
	\text{\textbf{Worksheet 16 Problem 3} }\\
	\text{Suppose that $\mathfrak{U} = (\vec{u}_1, \vec{u}_2 ,...,\vec{u}_n)$ is an orthonormal ordered basis for $\R^n$}
	\begin{center}
	$\begin{aligned}
	\text{[\,}\vec{x} \ \text{]\,}_\mathfrak{U}= \begin{bmatrix} \vec{x} \cdot \vec{u}_1 \\  \vec{x} \cdot \vec{u}_2 \\ \vdots \\ \vec{x} \cdot \vec{u}_n \\  \end{bmatrix}
	\end{aligned} $
	\end{center}
	\end{tcolorbox}


\pagebreak
\section{Orthogonality}
\begin{center} \end{center}

	\subsection{Orthogonal Complement Theorems}
	\begin{center} \end{center}
	
	\begin{tcolorbox}
	\text{\textbf{Book Theorem 5.4.1} }
	\begin{center}
	$\begin{aligned}
	(\operatorname{im}A)^\perp &= \operatorname{ker}(A^T) \\
	\operatorname{im}(A^T) &= (\operatorname{ker}A)^\perp
	\end{aligned} $
	\end{center}
	\end{tcolorbox}
	
	\begin{tcolorbox}
	\text{\textbf{Book Theorem 5.1.8}  }\\
	\text{If V is a subspace, then}
	\begin{center}
	$\begin{aligned}
	V^\perp  \text{ is a } & \text{subspace} \\
	(V^\perp)^\perp &= V \\
	n = \operatorname{dim}(V)  &+ \operatorname{dim}(V^\perp)  \\
	V \cap V^\perp &= \z \\
	\end{aligned} $
	\end{center}
	\end{tcolorbox}
	
	\begin{tcolorbox}
	\text{\textbf{Homework 7 Problem 4}  }\\
	\text{If V is a subspace, then}
	\begin{center}
	$\begin{aligned}
	V^\perp + W^\perp = (V \cap W)^\perp \\
	(V + W)^\perp = V^\perp \cap W^\perp 
	\end{aligned} $
	\end{center}
	\end{tcolorbox}
	
	\subsection{Orthogonal Matrix Characterizations}
	\begin{center} \end{center}
	
	\begin{tcolorbox}
	\text{\textbf{Book Theorem 5.3.8} }
	\begin{center}
	$\begin{aligned}
	A^TA&=I_n \\
	A ^{-1} &= A^T \\
	|| A\vec{x} || &= || \vec{x}|| \\
	(A\vec{x}) \cdot(A\vec{y}) &= \vec{x} \cdot \vec{y} \\
	\end{aligned} $
	\end{center}
	\end{tcolorbox}
	
	\begin{tcolorbox}
	\text{\textbf{Book Theorem 5.3.4} } \\
	\text{if $A$ and $B$ are orthogonal matrices}
	\begin{center}
	$\begin{aligned}
	AB & \text{ is orthogonal}\\
	A^{-1} & \text{ is orthogoanl}
	\end{aligned} $
	\end{center}
	\end{tcolorbox}
	
	\pagebreak
	\subsection{Graham Schmidt}
	\begin{center} \end{center}

	\begin{tcolorbox}
	\text{\textbf{Book Theorem 5.2.1} }\\
	\text{Construct an orthonormal basis $(\vec{u}_1, . . ., \vec{u}_n)$} \\
	\text{from any basis $(\vec{v}_1,  . . .,\vec{v}_n)$}
	\begin{center}
	$\begin{aligned}
	\vec{u}_j &= \frac{\vec{v}_j^\perp}{|| \vec{v}_j^\perp ||}\\
	\vec{v}_j^\perp = \vec{v}_j - (\vec{u}_1 \cdot \vec{v}_j)\vec{u}_1& - \cdots - (\vec{u}_{j-1} \cdot \vec{v}_j{j})\vec{u}_{j-1} \\
	\end{aligned} $
	\end{center}
	\end{tcolorbox}

	\subsection{QR Factorization}
	\begin{center} \end{center}
	
		\begin{tcolorbox}
	\text{\textbf{Book Theorem 5.2.2}  }\\
	\text{$QR$ factorization of a matrix $M$}
	
	\hfill
	\begin{center}
	$\begin{aligned}
	M  &= &Q \quad \quad \quad & \quad \quad \quad  \quad \quad \quad \quad R\\
	  \begin{bmatrix}  |&| & &| \\ \vec{b}_1 & \vec{b}_2 & \cdots & \vec{b}_n \\   |&| & &| \end{bmatrix} &= & \begin{bmatrix}  |&| & &| \\ \vec{u}_1 & \vec{u}_2 & \cdots & \vec{u}_n \\   |&| & &| \end{bmatrix}   & \begin{bmatrix} || \vec{v}_1 || & \vec{u}_1  \cdot \vec{v}_2 & \vec{u}_1  \cdot \vec{v}_3 & \cdots & \vec{u}_1  \cdot \vec{v}_n \\   0 &|| \vec{v}_2^\perp ||   &  \vec{u}_2  \cdot \vec{v}_3 & \cdots &  \vec{u}_2  \cdot \vec{v}_n \\ 0 & 0 &|| \vec{v}_3^\perp ||   & \cdots &  \vec{u}_3  \cdot \vec{v}_n \\ \vdots&\vdots&\vdots&\ddots&\vdots \\ 0 & 0 & 0 & \cdots & || \vec{v}_n^\perp ||  \end{bmatrix}\\
	\end{aligned}$
	\end{center}
	\end{tcolorbox}
	

\pagebreak
\section{Dot Product \& Projections}
\begin{center} \end{center}
	
	
	\subsection{Dot Product Theorems}
	\begin{center} \end{center}
	
	
	\begin{tcolorbox}
	\text{\textbf{Book Theorem 5.1.1} }
	\begin{center}
	$\begin{aligned}
	|| \vec{v} || &= \sqrt{\vec{v} \cdot \vec{v}} \\ 
	\vec{v} \cdot \vec{w} &= 0  \iff \vec{v} \perp \vec{w} \\
	\vec{u} \cdot \vec{u} & = 1 \iff \vec{u} \text{ is a unit vector}
	\end{aligned} $
	\end{center}
	\end{tcolorbox}
	
	\begin{tcolorbox}
	\text{\textbf{Book Theorem 5.3.6} }
	\begin{center}
	$\begin{aligned}
	A\vec{v} \cdot \vec{w} &= \vec{v} \cdot A^T\vec{w} 
	\end{aligned} $
	\end{center}
	\end{tcolorbox}
	
	\begin{tcolorbox}
	\text{\textbf{Book Theorem 5.1.9 - 5.1.12} }
	
\hfill	
	\begin{center}
	$\begin{aligned}
	||  \vec{x} +  \vec{y}||^2 = || \vec{x}||^2 + || \vec{y}||^2 \quad &\textbf{Pythagorean Theorem} \text{ *for orthogonal vectors} \\
	|| \text{proj}_V\vec{x}|| \leq || \vec{x}||\quad &\textbf{Magnitude of $ \text{proj}_V\vec{x}$} \\
	| \vec{x} \cdot \vec{y}| \leq || \vec{x}|| \  || \vec{y}||  \quad &\textbf{Cauchy-Schwarz Inequality} \\
	\cos \theta = \frac{\vec{x} \cdot \vec{y}}{|| \vec{x} ||\ || \vec{y} ||} \quad &\textbf{Angle between two vecors} \\
	\end{aligned} $
	\end{center}
	\end{tcolorbox}
	
	\subsection{Projections}
	\begin{center} \end{center}
		
	\begin{tcolorbox}
	\text{\textbf{Book Theorem 5.1.5}}\\
	\text{If $V$ is a subspace of $\R^n$ with orthonormal basis $\mathfrak{U} = (\vec{u}_1, \vec{u}_2 ,...,\vec{u}_n)$ then}
	
	\hfill
	\begin{center}
	$\begin{aligned}
	\text{proj}_V \vec{x} = (\vec{u}_1 \cdot\vec{x})\vec{u}_1 + (\vec{u}_2 \cdot\vec{x})\vec{u}_2 + \cdots + (\vec{u}_n \cdot\vec{x})\vec{u}_n
	\end{aligned} $
	\end{center}
	\hfill \text{for all $\vec{x}\in\R^n$}
	\end{tcolorbox}
	
	
	
	\subsection{Orthogonal Projection}
	\begin{center} \end{center}
	
	\begin{tcolorbox}
	\text{\textbf{Book Theorem 5.1.5}}\\
	\text{If $V$ is a subspce of $\R^n$ with orthogonal basis $\mathfrak{B} = (\vec{v}_1, \vec{v}_2 ,...,\vec{v}_n)$ then}
	\begin{center}
	$\begin{aligned}
	\text{proj}_V \vec{x} = \frac{\vec{v}_1 \cdot \vec{x}}{\vec{v}_1 \cdot \vec{v}_1}\vec{v}_1 + \frac{\vec{v}_2 \cdot \vec{x}}{\vec{v}_2\cdot \vec{v}_2}\vec{v}_2 + \cdots + \frac{\vec{v}_n \cdot \vec{x}}{\vec{v}_n \cdot \vec{v}_n}\vec{v}_n \end{aligned} $
	\end{center}
	\hfill \text{for all $\vec{x}\in\R^n$}
	\end{tcolorbox}
	
	
	\pagebreak
	\subsection{Matrix of an Orthogonal Projection}
	\begin{center} \end{center}
	
	
	\begin{tcolorbox}
	\text{\textbf{Book Theorem 4.3.4}  }\\
	\text{Projection onto a subspace $V \subseteq \R^n$ with orthonormal basis $\mathfrak{U} = (\vec{u}_1, \vec{u}_2 ,...,\vec{u}_d)$}\\
	\text{ and basis $\mathfrak{B} = (\vec{b}_1, \vec{b}_2 ,...,\vec{b}_d)$}
	
	\hfill
	\begin{center}
	$\begin{aligned}
	 P &= QQ^T \quad \text{where} \quad Q &=  \begin{bmatrix}  |&| & &| \\ \vec{u}_1 & \vec{u}_2 & \cdots & \vec{u}_d \\   |&| & &| \end{bmatrix} \\
	 P &= A(A^TA)^{-1}A^T  \quad \text{where} \quad A &= \begin{bmatrix}  |&| & &| \\ \vec{b}_1 & \vec{b}_2 & \cdots & \vec{b}_d \\   |&| & &| \end{bmatrix} \\
	\end{aligned}$
	\end{center}
	\end{tcolorbox}
	
	
	\begin{tcolorbox}
	\text{\textbf{Book Theorem 5.4.6}}\\
	\text{ $A\vec{x} = \vec{b}$ has the unique solution}
	\begin{center}
	$\begin{aligned}
	\vec{x}^*=(A^TA)^{-1}A^T\vec{b}
	\end{aligned} $
	\end{center}
	\end{tcolorbox}

	
\pagebreak
\section{Transpose  \& Least Squares}
\begin{center} \end{center}
	\subsection{Transpose Theorems}
\vspace{1em}
	\begin{tcolorbox}
	\text{\textbf{Book Theorem 5.3.9} }
	\begin{center}
	$\begin{aligned}
	(A + B)^T & = A^T + B^T \\
	(kA)^T &= kA^T   \\
	(AB)^T &= B^TA^T  \\
	\operatorname{rank}(A) &= \operatorname{rank}(A^T) \\
	(A^T)^{-1} &= (A^{-1})^T
	\end{aligned} $
	\end{center}
	\end{tcolorbox}
	
	\begin{tcolorbox}
	\text{\textbf{Homework 8 Problem 3} }
	\begin{center}
	$\begin{aligned}
	\operatorname{ker}(A) &= \operatorname{ker}(A^TA) \\
	\operatorname{im }(A) &= \operatorname{im}(AA^T) \\
	\end{aligned} $
	\end{center}
	\end{tcolorbox}
	
	\subsection{Least Squares Definition}
	\begin{center} \end{center}
	
	\begin{tcolorbox}
	\text{\textbf{Book Theorem 5.4.5}}\\
	\text{ $A\vec{x} = \vec{b}$ has the least-squares solutions of the consistent system}
	\begin{center}
	$\begin{aligned}
	A^T A \vec{x} = A^T \vec{b}
	\end{aligned} $
	\end{center}
	\end{tcolorbox}
	
	
	\begin{tcolorbox}
	\text{\textbf{Book Theorem 5.4.6}}\\
	\text{If ker($A$) = $\{\z\}$, then linear system}
	\hfill
	
	\text{ $A\vec{x} = \vec{b}$ has the unique solution}
	\begin{center}
	$\begin{aligned}
	\vec{x}^*=(A^TA)^{-1}A^T\vec{b}
	\end{aligned} $
	\end{center}
	\end{tcolorbox}
	
	\subsection{Least Squares Theorems}
	\begin{center} \end{center}
	
	\begin{tcolorbox}
	\text{\textbf{Book Theorem 5.4.6}}\\
	\text{If ker($A$) = $\{\z\}$, then linear system}
	\hfill
	
	\text{ $A\vec{x} = \vec{b}$ has the unique solution}
	\begin{center}
	$\begin{aligned}
	\vec{x}^*=(A^TA)^{-1}A^T\vec{b}
	\end{aligned} $
	\end{center}
	\end{tcolorbox}

	
\pagebreak
\section{Geometric Transformations}
\begin{center} \end{center}
	\subsection{Projections}
	\begin{center} \end{center}
	
	
	
	\subsection{Rotations}
	\begin{center} \end{center}
	
	\subsection{Reflections}
	\begin{center} \end{center}




	
	
	
\end{document}
	
	
	
	
	
	
	
	
	
	