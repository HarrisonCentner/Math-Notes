\documentclass[12pt]{article}
\usepackage{titlesec}
\usepackage{titling}
\usepackage[margin=.75in]{geometry}
\usepackage{pagecolor,lipsum}
\usepackage{xcolor}
\usepackage{fancyhdr}
\usepackage{multicol}
\usepackage{lmodern}
\pagecolor{white}
\usepackage{enumitem}
\usepackage{lineno}
\usepackage{parskip}
\usepackage{amssymb}
\usepackage{graphicx}
\usepackage{geometry}
\geometry{letterpaper}
\usepackage{amsmath}
\usepackage{mathrsfs}
\usepackage{array}
\usepackage{tcolorbox}

\linespread{1}

\pagestyle{fancy}

\newcommand*{\vertbar}{\rule[-1ex]{0.5pt}{2.5ex}}
\newcommand*{\horzbar}{\rule[.5ex]{2.5ex}{0.5pt}}

\titleformat{\section}%[frame]
{\vspace{-.15in}\LARGE\flushleft} %formatting
{} %numbering {\thesection}
{.0em} %distance between number and section title
{\bfseries \uppercase}[\titlerule] %Any code you want after gap before title

\titleformat{\subsection}
{\vspace{-1em}\Large\bfseries}
{\hspace{0em}}%$\bullet$}
{0em}
{\vspace{.3em}}[\vspace{-1.2em}]


\titleformat{\subsubsection}[runin] %runin eliminates in dent
{\vspace{0em}\bfseries}
{}
{0em}
{}[\ --- ]

\titlespacing{\subsubsection}
{0em}{-2em}{.5em} %Spacing of argument 1 left margin. 2 space before title. 3 space after title

\fancyhead{}

\renewcommand{\maketitle}{%renames a command
\begin {flushleft}{\fontsize{25pt}{33pt}\bfseries
\theauthor}}



% SHORTCUT COMMANDS
\newcommand{\p}{{\it{Proof.}}}
\newcommand{\z}{\vec{0}}
\newcommand{\R}{\mathbb{R}}
\newcommand{\qed}{\hfill \begin{flushright}$\square$\end{flushright}}
\newcommand{\B}{\mathcal{B}}
\newcommand{\hf}{\hfill}





%\rfoot{I made this r\'esum\'e with {\LaTeX}!}

\rhead{Introduction to Combinatorics — Math 465}
\renewcommand{\headrule}{\dotfill}


\begin{document}

\author{Enumerative Combinatorics Exam Notes}
\date{\today}
\fancyhead{}
\fancyhead[RO,RE]{Harry Centner — Prof. Terrence George — Exam One}

% \setbeamersize{text margin top=10mm}
\maketitle

\vspace{.7em} % adds a vertical space

\large{hcentner@umich.edu \hfill Harry Centner} \\
\hfill Professor Terrence George 


\end{flushleft}

\vspace{-1em}

\section{Coefficients}
\begin{center} \end{center}
\vspace{-1em}

	\subsection{Binomial Coefficients}
	\begin{center} \end{center}
	\begin{tcolorbox}
	\text{\textbf{Binomial Definitions }  }\\
	\text{For nonnegative integer $n$ and $k$,}
	
	\hfill
	\begin{center}
	$\begin{aligned}
	 \begin{pmatrix} n \\ k \end{pmatrix} = \frac{n!}{k!(n-k!)}\quad& \text{and}\quad \begin{pmatrix} n \\ k \end{pmatrix}  = 0 \quad \text{for $k >n$}\\
	 \hfill \\
	 \begin{pmatrix} n \\ k \end{pmatrix} &=  \begin{pmatrix} n \\ n-k \end{pmatrix} \\
	\end{aligned}$
	\end{center}
	\end{tcolorbox}
	
	
	\subsection{Multinomial Coefficients}
	\begin{center} \end{center}
	\begin{tcolorbox}
	\text{\textbf{Multinomial Definitions }  }\\
	\text{Note that nonnegative, integer $n = \sum_{i=1}^k m_i$,}
	
	\hfill
	\begin{center}
	$\begin{aligned}
	 \begin{pmatrix} n \\ m_1,\dots,m_k \end{pmatrix} &= \frac{n!}{m_1!\cdots m_k!}\\
	 \hfill \\
	 \begin{pmatrix} n \\ k \end{pmatrix} &=  \begin{pmatrix} n \\ k,n-k \end{pmatrix} \\
	\end{aligned}$
	\end{center}
	\end{tcolorbox}



	\subsection{Two Important Theorems}
	\begin{center} \end{center}

	\begin{tcolorbox}
	\text{\textbf{Binomial Theorem }  }\\
	\text{For nonnegative integer $n$,}
	
	\hfill
	\begin{center}
	$\begin{aligned}
	 (x+y)^n = \sum_{k=0}^n \begin{pmatrix} n \\ k \end{pmatrix}x^ky^{n-k} 
	\end{aligned}$
	\end{center}
	\end{tcolorbox}
	
	
	\begin{tcolorbox}
	\text{\textbf{Multinomial Theorem }  }\\
	\text{For nonnegative integer $n$,}
	
	\begin{center}
	$\begin{aligned}
	(x_1+\cdots + x_k)^n = \sum_{m_1+\cdots+m_k\ =\ n}^n  \begin{pmatrix} n \\ m_1, \dots , m_k \end{pmatrix}x_1^{m_1}\cdots x_k^{m_k}
	\end{aligned}$
	\end{center}
	\end{tcolorbox}






	\subsection{Binomial Identities}
	\begin{center} \end{center}
	
	
	\begin{tcolorbox}
	\text{\textbf{Pascal's Recurrence}  }\\
	\text{To construct Pascal's Triangle use,}
	\hfill
	$$   \begin{pmatrix} n \\ k \end{pmatrix} =   \begin{pmatrix} n-1 \\ k \end{pmatrix} +   \begin{pmatrix} n-1 \\ k-1 \end{pmatrix} $$
	\end{tcolorbox}
	
	

	\begin{tcolorbox}
	\text{\textbf{Binomial Relations}  }\\
	\text{For integer $n, \ m$, and $k$,}
	\hfill
	$$ \sum_{i=0}^n  \begin{pmatrix} n \\ i \end{pmatrix}  = 2^n$$
	$$ \sum_{i=0}^n (-1)^i  \begin{pmatrix} n \\ i \end{pmatrix}  = 0$$
	$$ \sum_{i=0}^l \begin{pmatrix} n \\ i \end{pmatrix} \begin{pmatrix} m \\ l-i \end{pmatrix} = \begin{pmatrix} n+m \\ l \end{pmatrix} $$
	$$  \sum_{i=0}^n   \begin{pmatrix} n \\ i \end{pmatrix} ^2 =    \begin{pmatrix} 2n \\ n \end{pmatrix}  $$
	$$  \sum_{j=0}^n  j \begin{pmatrix} n \\ j \end{pmatrix}  = n2^{2n-1}$$
	
	\text{For nonnegative integer $k$ and $n$ such that $k \le n$,}
	$$ \sum_{j=0}^n   \begin{pmatrix} j \\ k \end{pmatrix}  = \sum_{j=k}^n   \begin{pmatrix} j \\ k \end{pmatrix}  =  \begin{pmatrix} n+1 \\ k+1 \end{pmatrix}$$  

	\end{tcolorbox}

\newpage
\section{Generating Functions}
\begin{center} \end{center}
\vspace{-1em}
	
	\subsection{Binomial Coefficients}
	\begin{center} \end{center}
	
	\begin{tcolorbox}
	\text{\textbf{Binomial Definitions }  }\\
	\text{For integer $k$,}
	
	\hfill
	\begin{center}
	$\begin{aligned}
	 \begin{pmatrix} -n \\ k \end{pmatrix} &= (-1)^k  \begin{pmatrix} n +k-1\\ k \end{pmatrix} \\
	\end{aligned}$
	\end{center}
	\end{tcolorbox}
	
	
	
	\subsection{Generalized Theorems}
	\begin{center} \end{center}
	
	\begin{tcolorbox}
	\text{\textbf{Newton's Binomial Theorem }  }\\
	\text{For nonnegative integer $n$,}
	
	\hfill
	\begin{center}
	$\begin{aligned}
	 (x+y)^\alpha = \sum_{k=0}^\infty \begin{pmatrix} \alpha \\ k \end{pmatrix}x^ky^{\alpha-k} 
	\end{aligned}$
	\end{center}
	\hfill 
	\hfill \text{ holds whenever $y^{\alpha - k}$ and $x^b$ is uniquely defined.}
	\end{tcolorbox}
	
	\subsection{Standard Generating Functions}
	\begin{center} \end{center}
	
	\begin{tcolorbox}
	\text{\textbf{Fundamental Formal Power Series }  }\\
	\begin{center}
	$\begin{aligned}
	\frac{1}{1-x} &= \sum_{k=0}^\infty x^k \quad &h_n &= 1\\
	\frac{1}{1+x} &=  \sum_{k=0}^\infty (-1)^k x^k \quad& h_n &= (-1)^n\\
	 x\cdot\frac{d}{dx}\left(\frac{1}{1-x}\right) &=  \sum_{k=0}^\infty k x^k \quad &h_n &= n\\
	(1+x)^{n} &= \sum_{k=0}^\infty  \begin{pmatrix} n \\ k \end{pmatrix} x^k \quad &h_n &=  \begin{pmatrix} n \\ k \end{pmatrix}\\
	(1-x)^{-n} &= \sum_{k=0}^\infty  \begin{pmatrix} n+k-1 \\ n-1 \end{pmatrix} x^k \quad &h_n &=  \begin{pmatrix} n+k-1 \\ n-1\end{pmatrix}\\
	\end{aligned}$
	\end{center}
	\end{tcolorbox}
	
	\begin{tcolorbox}
	\text{\textbf{Finite Geometric Series }  }\\
	\text{For nonnegative integer $n$,}
	
	\hfill
	\begin{center}
	$\begin{aligned}
	\sum_{k=0}^n x^k= \frac{1-x^{n+1}}{1-x} 
	\end{aligned}$
	\end{center}
	\end{tcolorbox}
	
	
	\newpage
	\subsection{Weight Functions}
	\begin{center} \end{center}

	\begin{tcolorbox}
	\text{\textbf{Definition }  }\\
	\text{Let $\Omega: A\rightarrow \mathbb{Z}_{\ge0}$ be a "weight function" on a set $A$.}\\
	\text{For each value $k$, let}
	$$h_k = \text{ number of elements } a\in A \text{ with } \Omega(a) = k$$
	Then the generating function 
	$$H(x) = \sum_k h_k x^k = \sum_{a\in A} x^{\Omega(a)}.$$
	\hfill Note that $h(1) = |A|$.
	\end{tcolorbox}
	
	\begin{tcolorbox}
	\text{\textbf{Multiplication Principle }  }\\
	Suppose there is a bijection of the form 
	$$A \longleftrightarrow B\times C \times D\times \cdots$$
	$$a  \longleftrightarrow (b, c , d, \cdots).$$
	and that 
	$$\Omega: A\rightarrow \mathbb{Z}_{\ge0}$$
	$$\beta: B\rightarrow \mathbb{Z}_{\ge0}$$
	$$\gamma: C\rightarrow \mathbb{Z}_{\ge0}$$
	$$\delta: D\rightarrow \mathbb{Z}_{\ge0}$$
	$$\vdots$$
	are weight functions satisfying the additivity condition
	$$\Omega(a) = \beta(b) + \gamma(c) +\delta(d)+\cdots$$
	Then,
	$$\sum_{a\in A} x^{\Omega(a)} = \left( \sum_{b\in B} x^{\beta(b)} \right) \left(\sum_{c\in C} x^{\gamma(c)} \right) \left(  \sum_{d\in D} x^{\delta(d)}\right)\cdots$$
	\end{tcolorbox}
	
	
	

\newpage
\section{Linear Recurrences}
\begin{center} \end{center}
\vspace{-1em}

	\subsection{Homogenous Recurrences}
	\begin{center} \end{center}
	
	\begin{tcolorbox}
	\text{\textbf{Characteristic Expression}  }\\
	\text{For a homogenous linear recurrence of the form,}
	$$h_n + a_1h_{n-1} + a_2 h_{n-2} + \cdots + a_kh_{n-k} = 0$$
	\text{the characteristic expression is }
	$$q^k + a_1q^{k-1} + a_2a^{k-2} +\cdots + a_k = 0.$$
	\hfill \text{The roots and initial conditions gives the closed form of the recurrence.}
	\end{tcolorbox}
	
	\begin{tcolorbox}
	\text{\textbf{General Solution for Distinct Roots}  }\\
	\text{If the characteristic polynomial of a homogenous linear recurrence}\\
	\text{has distinct roots $\{q_1, \dots, q_k\}$, then its general solution is}
	$$h_n = \sum_{i=1}^k c_i q_i^n$$
	\end{tcolorbox}
	
	\begin{tcolorbox}
	\text{\textbf{General Solution for Multiplicitous Roots}  }\\
	\text{If a homogenous linear recurrence has characteristic polynomial}\\
	\text{with root $q_0$ of multiplicity $m$, then}
	$$h_n = n^w q_0^n \quad \text{for}\quad w\in \{0,1,\dots,m-1\}$$
	\hfill \text{satisfies the recurrence.} \\
	\text{In general, let $\{q_0,\dots,q_r\}$ be roots of the characteristic equation}\\
	\text{with respective multiplicities $\{m_0,\dots,m_r\}$, then  }\\
	$$h_n = C_1(n)q_0^n + \cdots + C_r(n)q_r^n$$
	\hfill \text{where each $C_i(n)$ is a polynomial in $n$ of degree less than $m_i$}
	\end{tcolorbox}
	
	\newpage
	\subsection{Non-homogenous Recurrences}
	\begin{center} \end{center}
	
	\begin{tcolorbox}
	\text{\textbf{General Solution}  }\\
	\text{Given a sequence $(b_n) = (b_0,b_1, b_2,\dots)$, the linear non-homogenous recurrence is}
	$$\sum_{i=0}^k a_i h_{n-i} = b_n$$
	\text{Strategy}
	\begin{enumerate}
	\item Find roots $\{ q_1,\dots,q_r\}$ of the characteristic expression for the homogenous recurrence.
	\item Guess solutions of the form $h_n = c\cdot b_n ,\  c_1\cdot n\cdot b_n,$ etc.
	\item Fit $c_1, \dots,c_r$ to initial conditions for $c_1(q_1)^n +c_2(q_2)^n + \cdots + c_r(q_r)^n + \{\text{GUESS}\}$
	\end{enumerate}
	\end{tcolorbox}
	
	\subsection{Difference Sequences}
	\begin{center} \end{center}
	
	\begin{tcolorbox}
	\text{\textbf{Difference Sequences of $h_n$ } }\\
	\text{The $k$-th difference sequence of $h_n$ is given by,}
	\begin{center}
	$\begin{aligned}
	\sum_{i=0}^k (-1)^i  \begin{pmatrix} k \\ i \end{pmatrix}h_{n-i}
	\end{aligned}$
	\end{center}
	\end{tcolorbox}

\begin{comment}
	\begin{tcolorbox}
	\text{\textbf{Discrete Taylor's Theorrem } }\\
	\text{The derivative is the shift operator,}
	\hfill
	\begin{center}
	$\begin{aligned}
	something
	\end{aligned}$
	\end{center}
	\end{tcolorbox}
\end{comment}

\newpage
\section{Special Counting Sequences}
\begin{center} \end{center}
\vspace{-1em}
	
	\subsection{Stirling Numbers}
	\begin{center} \end{center}
	
	\begin{tcolorbox}
	\text{\textbf{Stirling Numbers of the Second Kind } }\\
	\text{Are defined by,}
	
	\hfill
	\begin{center}
	$\begin{aligned}
	S(n,k) = \frac{1}{k!}\sum_{i=0}^k (-1)^i  \begin{pmatrix} k \\ i  \end{pmatrix} (k-i)^n &\quad \text{equivalently,} \quad S(n,k) = \frac{(-1)^k}{k!}\sum_{i=0}^k  (-1)^i   \begin{pmatrix} k \\ i  \end{pmatrix} i^n \\
	S(n,k) = S(n&-1,k-1)+k\cdot S(n,k-1)
	\end{aligned}$
	\end{center}
	\end{tcolorbox}
	
	\begin{tcolorbox}
	\text{\textbf{Signless Stirling Numbers of the First Kind  }}\\
	\text{Follow the recurrence relation,}
	
	\hfill
	\begin{center}
	$\begin{aligned}
	c(n,k) = c(n-1,k-1) + (n-1)\cdot c(n-1,k)
	\end{aligned}$
	\end{center}
	\end{tcolorbox}
	
	
	\begin{tcolorbox}
	\text{\textbf{Stirling Numbers of the First Kind  }}\\
	\text{Are defined by,}
	
	\hfill
	\begin{center}
	$\begin{aligned}
	s(n,k) = (-1)^{n-k} \cdot c(n,k)
	\end{aligned}$
	\end{center}
	\end{tcolorbox}
	
	\begin{tcolorbox}
	\text{\textbf{Formal Power Series Involving Stirling Numbers  }}\\
	\text{Stirling numbers provide closed forms for powers of $x$ and falling factorials,}
	$$x^n = \sum_{k=1}^n S(n,k)\cdot k! \begin{pmatrix} x \\ k  \end{pmatrix} = \sum_{k=1}^n S(n,k)\cdot (x)_k $$
	$$(x)_n = x(x-1)(x-2)\cdots(x-n+1) = n! \begin{pmatrix} x \\ n  \end{pmatrix}  =  \sum_{k=1}^n s(n,k) x^k$$
	\text{Interestingly, }
	$$	\begin{bmatrix} S(1,1) &  \cdots & 0  \\ \vdots  &\ddots &\vdots \\ S(n,1)& \cdots  & S(n,k)\end{bmatrix} \begin{bmatrix} s(1,1) &  \cdots & 0 \\ \vdots & \ddots &\vdots  \\ s(n,1)&\cdots  & s(n,k)\end{bmatrix} = I_n$$
	\end{tcolorbox}
	
	
	
	\begin{tcolorbox}
	\text{\textbf{This Sequence Counts \dots}  }\\
	\text{$k!\cdot S(n,k)$ enumerates: }
	\begin{itemize}
	\item Partitions of $\{1,\dots,n\}$ into $k$ distinguished, non-empty boxes.
	  \subitem Divide by $k!$ for identical boxes.
	\item Surjections $f:\{ 1,\dots, n\}\rightarrow\{1,\dots,k\}$.
	\end{itemize}
	
	\text{$c(n,k)$ enumerates: }
	\begin{itemize}
	\item Permutations of  $\{1,\dots,n\}$ with $k$  disjoint cycles.
	\end{itemize}
	\end{tcolorbox}
	
	
	
	\subsection{Fibonacci Numbers}
	\begin{center} \end{center}
	
	\begin{tcolorbox}
	\text{\textbf{Definition}  }\\
	\text{For nonnegative integer $n$, }
	\hfill
	$$f_n = \frac{1}{\sqrt{5}}\left(\frac{1 + \sqrt{5}}{2}\right)^n - \frac{1}{\sqrt{5}}\left(\frac{1 - \sqrt{5}}{2}\right)^n $$
	$$f_n = f_{n-1} + f_{n-2} $$
	\end{tcolorbox}
	
		
	\begin{tcolorbox}
	\text{\textbf{This Sequence Counts \dots}  }\\
	\text{$f_{n+1}$ enumerates: }
	\begin{itemize}
	\item Binary strings of length $n-1$ that do not contain consecutive 1s.
	\item Representations of $n$ as an ordered sum of 1s and 2s.
	\item Tilings of a $2\times n$ board with dominoes.
	\end{itemize}
	\end{tcolorbox}
	
	
	\subsection{Derangement Numbers}
	\begin{center} \end{center}
	
	\begin{tcolorbox}
	\text{\textbf{Definition}  }\\
	\text{For nonnegative integer $n$, }
	\hfill
	$$D_n = n\sum_{k=0}^n \frac{(-1)^k}{k!} $$
	$$D_n = n\cdot D_{n-1} + (-1)^n$$
	$$D_n = \left\{ \text{closest integer to } \frac{n!}{e} \right\}$$ 
	\end{tcolorbox}
	
	\begin{tcolorbox}
	\text{\textbf{This Sequence Counts \dots}  }\\
	\text{$D_n$ enumerates: }
	\begin{itemize}
	\item Permutations of $\{1,\dots,n\}$ with no one-cycles. 
	\item Equivalently, permutations of $\{1,\dots,n\}$ with no number in its natural position.
	\end{itemize}
	\end{tcolorbox}
	
	\newpage
	\subsection{Catalan Numbers}
	\begin{center} \end{center}
	
	\begin{tcolorbox}
	\text{\textbf{Definition}  }\\
	\text{For nonnegative integer $n$, }
	\hfill
	$$C_n = \frac{1}{n+1}\begin{pmatrix} 2n \\ n \end{pmatrix} = \frac{1}{2n+1}\begin{pmatrix} 2n+1 \\ n\end{pmatrix}   $$
	$$C_n = \sum_{k+l = n-1}^{n-1} C_k\cdot C_l\quad \text{equivalently}\quad C_{n+1} = \sum_{k=0}^n C_k \cdot C_{n-k}$$
	\end{tcolorbox}
	
	
	\begin{tcolorbox}
	\text{\textbf{This Sequence Counts \dots}  }\\
	\text{$C_n$ enumerates: }
	\begin{itemize}
	\item Ballot sequences of length $2n$.
	\item Dyck Paths from $(0,0)$ to $(2n,0)$ above $y=0$.
	\item Lattice Paths from $(0,0)$ to $(n,n)$ above $y=x$.
	\item Triangulations of a convex $(n+2)$-gon by non-crossing diagonals.
	\item Ordered trees on $n+1$ unlabeled vertices.
	\item Complete (binary) ordered trees with $n+1$ leaves (or equivalently $n$ internal vertices).
	\item Syntactically correct bracketings of a sequence of $n+1$ letters.
	\end{itemize}
	\end{tcolorbox}
	
	
	\subsection{Partition Numbers}
	\begin{center} \end{center}
	
	\begin{tcolorbox}
	\text{\textbf{Definition}  }\\
	\text{For nonnegative integer $k$, }
	\hfill
	$$\sum_{k=1}^\infty p(k)x^k = \prod_{k=1}^\infty \frac{1}{1-x^k}$$
	$$\sum_{k=1}^\infty p_d(k)x^k = \prod_{k=1}^\infty (1+x^k)$$
	\end{tcolorbox}
	
	
	\begin{tcolorbox}
	\text{\textbf{This Sequence Counts \dots}  }\\
	\text{$p(n)$ enumerates: }
	\begin{itemize}
	\item All weakly decreasing strong compositions of $n$.
	\item All (unordered) multisets of positive integers whose sum is $n$.
	\item Nonnegative integer solutions of $m_1+2m_2+3m_3+\cdots = n$.
	\end{itemize}
	\text{$p_d(n)$ enumerates: }
	\begin{itemize}
	\item Partitions with distinct parts.
	\end{itemize}
	\end{tcolorbox}
	
	
	
\newpage
\section{Common Things to Count}
\begin{center} \end{center}
\vspace{-1em}
	

	\subsection{Enumerating Sets with Intersections}
	\begin{center} \end{center}
	
	\begin{tcolorbox}
	\text{\textbf{Inclusion-Exclusion Principle}}\\
	\text{Let $S$ be a finite set and $A_1,\dots,A_r \subseteq S$.}\\
	\text{Then,}
	$$\left|S - \bigcup_{i=1}^r A_i \right| =  \sum_{i=1}^r (-1)^i n_i $$
	 \text{where $n_0 = |S|$ and, for $i \in \{1,\dots,r\}$}\\
	$$n_i = \sum_{1\le j_1<\cdots<j_i\le} \left| A_{j_1}\cap \cdots \cap A_{j_i}\right|$$
	\hfill \text{the sum ranges over all subsets of $A_1, \dots, A_r$.}\\
	\end{tcolorbox}
	
	
	\begin{tcolorbox}
	\text{\textbf{Euler's Totient Function}}\\
	\text{Let $\{p_1,\dots,p_r\}$ be the set of distinct prime divisors of $n$. }\\
	\text{Let $\phi: \mathbb{Z}_{>0} \rightarrow \mathbb{Z}_{>0}$ be the mapping from $n$ to the number of integers coprime to $n$.}\\
	\text{Then,}
	
	\hfill
	\begin{center}
	$\begin{aligned}
	\phi(n) = n\prod_{j=1}^r\left(1-\frac{1}{p_j}\right)
	\end{aligned}$
	\end{center}
	\end{tcolorbox}
	

	\subsection{Compositions}
	\begin{center} \end{center}
	
	\begin{tcolorbox}
	\text{\textbf{Weak Compositions}  }\\
	\text{Enumerates the nonnegative integer solutions of $x_1+\cdots+x_k = n$ is}
	\hfill
	$$ \begin{pmatrix} n+k-1 \\ k-1 \end{pmatrix}   $$
	\end{tcolorbox}
	
	\begin{tcolorbox}
	\text{\textbf{Strong Compositions}  }\\
	\text{Enumerates the positive integer solutions of $x_1+\cdots+x_k = n$ is}
	\hfill
	$$\begin{pmatrix} n-1 \\ k-1 \end{pmatrix} $$
	$$\sum_{k=1}^n \begin{pmatrix} n-1 \\ k-1 \end{pmatrix} = 2^{n-1}$$
	\end{tcolorbox}
	
	\newpage
	\subsection{Lattice Paths}
	\begin{center} \end{center}
	
	\begin{tcolorbox}
	\text{\textbf{Set Partitions}  }\\
	\text{Enumerates the paths from $(0,0)$ to $(n,m)$ with moves $(1,0)$ and $(0,1)$}
	\hfill
	$$\begin{pmatrix} n+m \\ n \end{pmatrix} \quad \text{equivalently} \quad \begin{pmatrix} n+m \\ m\end{pmatrix} $$
	\end{tcolorbox}
	
	
	
	\subsection{Subsets}
	\begin{center} \end{center}
	
	\begin{tcolorbox}
	\text{\textbf{Set Partitions}  }\\
	\text{Enumerates the number of subsets of a set $A$ such that $|A| = n$}
	\hfill
	$$2^n$$
	\end{tcolorbox}
	
	\subsection{Finite Strings}
	\begin{center} \end{center}
	
	\begin{tcolorbox}
	\text{\textbf{Alphabets}  }\\
	\text{The number of $k$-digit strings over an $n$-element alphabet is}
	\hfill
	$$n^k$$
	\text{The number of $k$-digit strings over an $n$-element alphabet,}\\
	\text{in which no letter is used more than once, is}
	$$(n)_k= \frac{n!}{(n-k)!}$$
	\end{tcolorbox}


	
\end{document}